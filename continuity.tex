\documentclass[a4paper]{article}

% Assumes A4 paper
%
\setlength{\textheight}{20cm}
\setlength{\textwidth}{16cm}
\setlength{\hoffset}{-1.0cm}
\setlength{\footskip}{2.5cm}
\setlength{\headsep}{1.5cm}
\setlength{\voffset}{-1.5cm}
\usepackage{amsmath}
%\usepackage{url}
\usepackage{graphicx}

\title{Technical note: continuity matrix for estimating slope-sensor 
noise and reduction thereof for algorithm-based reconstruction}
\author{Nazim Ali Bharmal\\
n.a.bharmal@durham.ac.uk}
\date{\today}

\begin{document}
\maketitle

\section{Overview}

\begin{itemize}
   \item {\bf AIM:} To show that noise in a measurement of slopes with a WFS
   that is noise-uncorrelated can be estimated using the prior that wavefront
   is continuous but slopes are not necessarily so, and then removed using the
   same approach as estimation, with extension to sparse representation.

   \item {\bf WHY:} This work was done in the context of noise reduction for
   HWR\footnote{a.k.a. DiCuRe}. Later it was understood that this would offer a
   method for measurements with elongation in a SH which leads to
   correlated-noise, and so a potential on-line method for estimating noise
   covariance matrices.
   
   \item {\bf IMPORTANT:} Much was noted in a 1980 paper\footnote{J. Herrmann.
   Least-squares wave front errors of minimum norm. J. Opt. Soc. Am.,
   70(1):28–35, Jan 1980.} but new here is the extension to correlated-noise and
   sparse representation.

   \item {\bf RESULTS:} It works, the non-continuous noise estimated can be
   quantified as a sparsity-dependent fraction of the total noise, but with
   constant variance so SNR of noise estimation is invesely proportional
   to the sparsity. Since the direct application of continuity is sparse
   from the outset, clearly the un-sparse option is best.
   For noise removal, the required inversion then means a trade-off between
   sparsity and quantity eliminated leads to an optimum sparsity of {\bf ZZZ}.

   \item {\bf OUTSTANDING:} No work done on the correlated-noise, the strategy
   at present is assumed to be based on regularization in a least-squares
   inversion.
\end{itemize}

\subsection{Unresolved issues}

\begin{enumerate}
  \item {\bf .}  
\end{enumerate}

\section{Introduction}

From a slope WFS which is the current type of interest with modern AO, there is
associated noise per slope per sub-aperture. This noise may be uncorrelated,
for example when the spot shape is round so that there is no difference is
orthogonal spot motion. Alternately, if the spots have structure such as
elongation then their motion will have a smaller signal in the directions
of maximum elongation and so effectively larger noise in that direction. In 
this latter case, noise on slopes is uncorrelated along and perpendicular
to the elongation direction and so instead does become correlated.

It was pointed by Herrmann (1980) that since a wavefront is continuous,
the slopes must be correlated in the two orthogonal directions regardless
of the underlying statistical description of the wavefront. Slope noise
however does not require this property. Knowledge of this difference can allow
for a straightforward estimate of the slope noise since integration of slopes
about 

if a model is made equivalent to the
Fried geometry, such that 


A conventional interaction (a.k.a. poke) matrix is formed by,

\begin{equation}
   \mathnormal{P}=\mathnormal{R}+\mathnormal{\epsilon},
\end{equation}

where $\mathnormal{P}$ is the interaction matrix and $\mathnormal{R}$ is the
measurement matrix, with $\mathnormal{\epsilon}$ being the noise so that the
quantity used is the sum of these last two terms. This equation is true only
when the actuator command vectors (that can form a matrix, $\mathnormal{V}$) are
set so that only one has a non-zero value per column of the matrix
i.e.~$\mathnormal{V}\equiv V_{amp}\mathnormal{I}$. More generally it can be written
that,

\begin{equation}
   \mathnormal{G}\mathnormal{V}=\mathnormal{R}+\mathnormal{\epsilon},
   \label{eqn:gen}
\end{equation}

where now our goal is to discover $\mathnormal{G}$ which is the matrix that
describes the response of the wavefront sensor for a given actuator command
vector. Subsequently the command matrix (not, technically, the reconstruction
matrix) called $\mathnormal{G}^{\dagger}$ can be derived as,

\begin{equation}
   \mathnormal{G}^{\dagger}=\mathnormal{V}\mathnormal{R}^{\dagger}.
   \label{eqn:gplus}
\end{equation}

The noise means that,

\begin{equation}
   \mathnormal{G}^{\dagger}\rightarrow\mathnormal{V}\left(
      \mathnormal{R}+\mathnormal{\epsilon}\right)^{\dagger}.
   \label{eqn:noisygplus}
\end{equation}

\section{Theory}

To minimize the effect of noise, instead consider finding
$\mathnormal{G}=\mathnormal{R}\mathnormal{V}^{\dagger}$ leading to,

\begin{eqnarray}
   \mathnormal{G'}&=&
      \left(\mathnormal{R}+\mathnormal{\epsilon}\right)\mathnormal{V}^{\dagger},\\
   \delta\mathnormal{G}&=&\mathnormal{\epsilon}\mathnormal{V}^{\dagger}.
\end{eqnarray}

Then it can be said an optimal set of measurements ($\mathnormal{R}$) results in a
`maximum' value for $\mathnormal{G}$ relative to $\delta\mathnormal{G}$ such that
$\mathnormal{G'}$ is as close to $\mathnormal{G}$ as possible.

An appropriate characterization of $\mathnormal{G}$ is the operator norm,

\begin{equation*}
   \|\mathnormal{A}\|_{o}= \|\mathnormal{A}\mathnormal{x}\|_{2},
\end{equation*}

where $\mathnormal{x}$ is an arbitrary vector (usually with 2-norm equal to one)
and the 2-norm is defined for a vector by $\|x\|_{2}=x.x$. Then it can stated
that the requirement for minimum effect of noise (for a fixed noise level) is
to find $\mathnormal{R}$ ($\mathnormal{V}$ is dependent only on this matrix) such that,

\begin{equation*}
     {\left\|\mathnormal{R}\mathnormal{V}^{\dagger}\right\|_{o}}
     \left/
     {\left\langle
      \left\|\mathnormal{\epsilon}\mathnormal{V}^{\dagger}\right\|_{o}
      \right\rangle}
     \right.=
   \sigma_\epsilon^{-2}\,\mathrm{max}\left\{
     {\left\|\mathnormal{R}\mathnormal{V}^{\dagger}\right\|_{o}}
     \left/
     {\left\|\mathnormal{V}^{\dagger}\right\|_{o}}
     \right.
   \right\}.
\end{equation*}

To proceed further, the Hilbert-Schmidt norm and its relation to the operator
norm is introduced,

\begin{eqnarray*}
   \|\mathnormal{A}\|_{HS}&=& \left( \sum_{i,j} A_{ij}^2 \right)^{1/2},\\
   &&\|\mathnormal{A}\|_{o}\le\|\mathnormal{A}\|_{HS}\le\sqrt{2}N\|\mathnormal{A}\|_{o},\\
\end{eqnarray*}

where $2N^{2}$ is the number of entries in $\mathnormal{A}$, so the HS norm is
bound. (For AO interaction matrices, $N$ is the number of actuators and so this
is why this terminology is used here.) A further useful relation is that,

\begin{equation*}
   \|\mathnormal{A}\mathnormal{B}\|_{HS}=\|\mathnormal{A}\|_{HS}\|\mathnormal{B}\|_{HS}.
\end{equation*}

So it is {\it postulated} that the desired maximization is instead of, and
assuming that $\mathnormal{V}^{\dagger}$ does not span the null-space of
$\mathnormal{R}$,

\begin{eqnarray}
     {\left\|\mathnormal{R}\mathnormal{V}^{\dagger}\right\|_{HS}}
     \left/
     {\left\|\mathnormal{V}^{\dagger}\right\|_{HS}}.
     \right. &=&
   \mathrm{max}\left\{
     {\left\|\mathnormal{R}\mathnormal{V}^{\dagger}\right\|_{HS}}
     \left/
     {\left\|\mathnormal{V}^{\dagger}\right\|_{HS}}
     \right.
   \right\} \nonumber \\
     &=&
   \sigma_\epsilon^{-1}\mathrm{max}\left\{
     {\left\|\mathnormal{R}\right\|_{HS}} \right\} \label{eqn:cond2} .
\end{eqnarray}

In other words, the na\"{i}ve statement that the measurement matrix
$\mathnormal{R}$ which has highest SNR is that with entries that are as large as
possible, does indeed lead to the solution outlined in equation
\ref{eqn:cond2}. In the ideal scenario, $\mathnormal{R}$ would contain entries
which correspond to positive/negative limits of the WFS range.

The contrast with Kasper {\it et~al.} is that their requirement is that
$\mbox{det}\left\{\mathnormal{V}\right\}$ should be maximized and the condition
number of the same matrix be equal to one.

\section{Approach}

To obtain the maximal measurement matrix, while this is more straightforward
for WFSs that have a 1:1 mapping between sub-aperture and actuator command,
this is not the case for slope sensors where there are (in the Fried geometry)
2 slopes dependent on 4 actuators. Furthermore, there are possible
arrangements of the actuator commands which result in no signal and are so in
the null space of $\mathnormal{G}$ e.g.~diagonally opposite actuators (forming a
pair) having equal commanded values while each pair has different command
values. The interest here is to form $\mathnormal{R}$ and then infer $\mathnormal{V}$.

The approach is to specify half of the slopes in $\mathnormal{R}$ to begin with,

%\begin{math}
\begin{displaymath} 
\mathnormal{R} = \left(
   \begin{array}{ccc} 
      R_{11} & \ldots & R_{1N} \\ 
      \vdots & \ddots & \vdots \\ 
      R_{m/2\,1} & \ldots & R_{m/2\,N}  \\\hline
      0 & \ldots & 0 \\ 
      \vdots & \ddots & \vdots \\ 
      0 & \ldots & 0 
   \end{array} \right) = \left[
   \begin{array}{c} 
      \mathnormal{R}_U \\ 0%\mathnormal{R}_L
   \end{array} \right]
\end{displaymath} 
%\end{math}

Let it be assumed that the range limit is $\pm1$. Then given a fully populated
$\mathnormal{R}$, wavefront continuitity implies,

\begin{equation}
   \mathnormal{L}\mathnormal{R}=0=\left[
      \mathnormal{L}_U\,\mathnormal{L}_L
   \right]\left[
   \begin{array}{c} 
      \mathnormal{R}_U \\ \mathnormal{R}_L
   \end{array} \right],
\end{equation}

where $\mathnormal{L}$ is the loop integration operator. This comes from noise
reduction in HWR and is a matrix that integrates loops of
slopes\footnote{discrete curl, I think.}. Returning to a half-populated
$\mathnormal{R}$, assuming the first (upper) set of slopes, then $\mathnormal{L}$
can also be split and the following written,

\begin{eqnarray}
   \mathnormal{L}_U\mathnormal{R}_U+\mathnormal{L}_L\mathnormal{R}_L&=&0
      \nonumber \\
%   \mathnormal{L}_L\mathnormal{R}_L&=&-\mathnormal{L}_U\mathnormal{R}_U \nonumber \\
   \mathnormal{R}_L&=&-\mathnormal{L}_L^{\dagger}\mathnormal{L}_U\mathnormal{R}_U,
\end{eqnarray}

and $\mathnormal{R}_L$ found to form $\mathnormal{R}$. This does not guarantee
that $\mathnormal{R}_L$ consists of entries that are bounded to $\pm1$, so
further steps are required. These are somewhat {\it adhoc} but do seem to
produce a decent result.

\subsection{Method used to achieve a suitable $\bf{R}$}

\begin{enumerate}
   \item Take the intial estimate for $\mathnormal{R}$ and binarize it such that,
   \begin{equation*}
      \mathnormal{R}_b=\mathrm{binarize}\left\{\mathnormal{R}\right\}=
      \left\{ \begin{array}{ll}
         -1 & (R_{ij}+\nu_{ij}) < 0 \\
         +1 & (R_{ij}+\nu_{ij}) > 0,
         \end{array} \right.
   \end{equation*}
    noting that the matrix $\nu$ consists of infintesimally small random values
    to ensure exactly zero values are randomly made negative or positive.
   \item This matrix will not have slope values consistent with wavefront
   continuity, so an estimate can be made of the correction to slopes
   which does ensure continuity,
   \begin{equation*}
      \mathnormal{R}_c=\left(\mathnormal{I}-\mathnormal{L}^{\dagger}\mathnormal{L}\right)
         \mathrm{binarize}\left\{ \mathnormal{R} \right\}.
   \end{equation*}
   \item The modified measurement $\mathnormal{R}_c$ is then not guaranteed to have
   gradients which are limited to the range $\pm1$.
   \item One approach that does converge on a stable and consistent form in both
   $\mathnormal{R}$ and $\mathnormal{V}$ is to iterate this procedure,
   \begin{itemize}
      \item $i=0$
      \item \textbf{while}
       $\mbox{var}\left\{\left|\mathnormal{R}_c\right|-1\right\}<\mbox{tolerance}$
       \textbf{or} $i < 1000$
      \begin{itemize}
         \item $\mathnormal{R}_c=\left(\mathnormal{I}-\mathnormal{L}^{\dagger}
            \mathnormal{L}\right) \mathrm{binarize}\left\{ \mathnormal{R}_c \right\}$
         \item $i=i+1$
      \end{itemize}
   \end{itemize}
   \item Examples of the final $\mathnormal{R}_c$ matrix are shown in figure
     \ref{fig:iter}.
\end{enumerate}

\begin{figure} \begin{center}
   \includegraphics[width=12cm,angle=0]{hadamard_iterHadamardOrigin.png} 
   \includegraphics[width=12cm,angle=0]{hadamard_iterRandomOrigin.png} 
   \caption{ Examples of the iterative procedure for calculating
    $\mathnormal{R}_c$ with different starting values for $\mathnormal{R}_U$.
    The top plot is the resulting actuator command matrix, $\mathnormal{V}$,
    bottom left is $\mathnormal{R}_c$, and the histogram shows the distribution
    of $\mathnormal{R}_c$ (here $\pm0.5$).
   {\it Top:} $\mathnormal{R}_U=\mathnormal{G}\mathnormal{H}_{256}$,
   {\it Bottom:}  Starting with $\mathnormal{R}_U=\eta$, random intial entries.
  \label{fig:iter} }
\end{center} \end{figure} 

\section{Results}

There are now three alternative formations for $\mathnormal{R}$. First, the
conventional poke matrix which was not intended to be the most accurate
when noise is considered. Second, the approach of Kaspar {\it et~al.} 
where $\mathnormal{V}$ is defined as a Hadamard matrix. Third, the approach
outlined here. To use a Hadamard matrix in most situations, they are only
easily defined for matrices with rank that is a power of 2. (Hadamard
matrices are not known to exist for all ranks between 1 and 100.) Equation
\ref{eqn:gen} can be modified for cases where $\mathnormal{V}$ has larger rank
than required given the number of actuators, and so is replaced with a
Hadamard matrix, $\mathnormal{H}$, 

\begin{eqnarray}
   \left[ G\;0 \right ]\mathnormal{H}&=&\left[\begin{array}{c}
      \mathnormal{R} \\
      0 \end{array} \right] \nonumber \\
   \left[ \mathnormal{G}^{\dagger}\;0 \right ]&=&\left(
      \left.\left[\begin{array}{c}
      \mathnormal{R} \\
      0 \end{array} \right]\mathnormal{H}^{T}\right/
         \mathrm{rank}\left\{\mathnormal{H}\right\}
      \right)^{\dagger},
\end{eqnarray}

where the orthogonality of $\mathnormal{H}$ is utilized,
$\mathnormal{H}^{T}\mathnormal{H}=\mathrm{rank}\left\{\mathnormal{H}\right\}\mathnormal{I}$. To
form of $\mathnormal{G}^{\dagger}$ for the other cases, equation \ref{eqn:gplus}
can be used. The norms for $\mathnormal{R}$ for the three cases are shown in table
\ref{table:hsnorm}. The resulting differences between the diagonal actuator
command matrix response and the reconstruction using each form of
$\mathnormal{G}^{\dagger}$ from equation \ref{eqn:noisygplus} are shown in
\ref{table:snrres} for two SNR cases.

The Hadamard matrix as an extended actuator control matrix had a large inherent
inaccuracy although the response to noise was similar to the use of the maximum
value measurement matrix. The maximum value measurement approach resulted in
the least sensitivity to noise, with the conventional poke matrix being
approximately $2\times$ worse except at SNR equalling 3. This latter result
implies a lower accuracy and could be the result of using SVD in this work
rather than a more sophisticated approach.

\begin{table}
\begin{center}\begin{tabular}{r|c|c|c} \hline
   \textbf{model}             & \multicolumn{3}{c}{
      $\mbox{var}\left\{\mathnormal{G}^{\dagger}\mathnormal{R}-\mathnormal{I}\right\}$}
                                                                        \\ \hline
            & SNR=$10^{3}$       & SNR=$10^{2}$       & SNR=10          \\ \hline
   Poke     & $1.3\times10^{-6}$ & $1.4\times10^{-4}$ & $3.1\times10^{-3}$ \\
   Kaspar   & $4.7\times10^{-3}$ & $4.7\times10^{-3}$ & $4.9\times10^{-3}$ \\
   Max $\mathnormal{R}$, Hadamard 
            & $2.5\times10^{-6}$ & $9.9\times10^{-5}$ & $1.4\times10^{-3}$ \\
            \hline 
\end{tabular}\end{center}
   \caption{Error when gaussian noise is added to the measurement matrix forming
   the control matrix.
   \label{table:snrres}}
\end{table}

\begin{figure} \begin{center}
   \includegraphics[width=12cm,angle=0]{hadamard_gplusMs.png} 
   \caption{.
  \label{fig:gplusms}}
\end{center} \end{figure} 

\begin{table}
\begin{center}\begin{tabular}{r|c} \hline
   \textbf{model}             & \textbf{HS norm} \\ \hline
   Poke                       & 40 \\
   Kaspar                     & 147 \\
   Max $\mathnormal{R}$, random   & 293 \\
   Max $\mathnormal{R}$, Hadamard & 293 \\
   (Maximum norm)             & 297 \\ \hline 
\end{tabular}\end{center}
   \caption{The Hilbert-Schmidt norms for the 3 different approaches to
   forming $\mathnormal{R}$, with the maximum theoretical norm.
   \label{table:hsnorm}}
\end{table}

\end{document}
